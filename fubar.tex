\documentclass[12pt,a4paper]{article}
\usepackage[utf8]{inputenc}
\usepackage{a4wide}
\usepackage{geometry}
\usepackage{amsmath}
\usepackage{ellipsis}
\geometry{vmargin=2.5cm,hmargin=2cm}
\author{Marvin Cohrs}
%\institute{Kopernikus-Gymnasium, Rheine}
\title{Fu Bar}

\begin{document}

	\maketitle
	\tableofcontents
	
	\section{Skills}
	\subsection{Operators + Functions}
	
	Fu Bar knows several operators and functions:
	
	\begin{itemize}
		\item Addition: $x + y$\\
			  Adds two floating-point numbers.\\
			  e.g. $4.1 + 5.8 = 9.9$
		\item Subtraction: $x - y$\\
			  Subtracts a floating-point number from another one.\\
			  e.g. $9.9 - 5.8 = 4.1$
		\item Multiplication: $x * y$\\
			  Multiplies two floating-point numbers.\\
			  e.g. $1.1 * 9.0 = 1.1 \times 9.0 = 9.9$
		\item Division: $x / y$\\
			  Divides a floating-point number by another one.\\
			  e.g. $9.9 / 1.1 = \frac{9.9}{1.1} = 9.0$
		\item Integer division: $n \backslash m$\\
			  Divides an integer by another one and returns the integer quotient.\\
			  e.g. $20 \backslash 6 = 3$\\
			  because $6\times3=18$ and $6\times4=21$
		\item Modulo: $n \# m$\\
			  Divides an integer by another one and returns the modulo.\\
			  e.g. $20 \# 6 = 2$\\
			  because $6\times3+2=20$
		\item Exponentiation: $x \; \hat{ } \; n$\\
			  Exponentiates a floating-point base by an integer exponent.\\
			  e.g. $2 \; \hat{ } \; 8 = 2 ^{8} = 256$
		\item Paranthenses: $(expression)$\\
			  An expression in paranthenses has got a higher priority.\\
			  e.g. $2 * (4 + 5) = 2 * 9 = 18$\\
			  but $2 * 4 + 5 = 8 + 5 = 13$
		\item Absolute Value: $|expression|$, $abs(expression)$\\
			  Gets the absolute value of an expression.\\
			  e.g. $|+3| = 3$\\
			  and $|-3| = 3$, too!
		\item Negation: $-x$, $neg(x)$\\
		      Negates a floating-point number.\\
		      e.g. $- (3 * 2) = -6$
		\item Reciprocal: $1/x$, $rec(x)$\\
			  Divides $\frac{1}{x}$, resulting the reciprocal.\\
			  e.g. $rec(4) = \frac{1}{4} = 0.25$
		\item Factorial: $n!$, $fac(n)$\\
			  Computes the factorial of an integer.\\
			  e.g. $4! = 24$
		\item Binomial Coefficient: $k$ from $n$, $n$C$k$\\
			  Computes the binomial coefficient for $n \choose k$\\
			  e.g. $2$ from $3 = 3$C$2 = {3 \choose 2} = 3$
		\item Square root: $sqrt(x)$, $SquareRoot(x)$\\
			  Computes the square root of a floating-point number.\\
			  e.g. $sqrt(9) = \sqrt{9} = 3$
		\item Cube root: $cbrt(x)$, $CubeRoot(x)$\\
			  Computes the cube root of a floating-point number.\\
			  e.g. $cbrt(27) = \sqrt[3]{27} = 3$
		\item $n$th root: $nrt(x; n)$, $Root(x; n)$\\
			  Computes the $n$th root of a floating-point number.\\
			  e.g. $nrt(4096; 6) = \sqrt[6]{4096} = 4$
		\item Integer reduction: $int(x)$, $x\backslash1$\\
			  Reduces a floating-point number to an integer:\\
			  e.g. $int(tau) = 6$
		\item Cut: $frac(x)$\\
			  Cut's the number at the floating-point, resulting fraction part.\\
			  e.g. $frac(23.45) = 0.45$
		\item Binary conjunction: $n$ and $m$\\
			  Conjuncts two integers.\\
			  e.g. $7$ and $11 = 3$
		\item Binary disjunction: $n$ or $m$\\
			  Disjuncts two integers.\\
			  e.g. $7$ or $11 = 15$
		\item Binary contravalence: $n$ xor $m$\\
			  Contravalences two integers.\\
			  e.g. $7$ or $11 = 12$
		\item Greatest common divisor: $gcd(n; m)$\\
			  Computes the greatest common divisor of two integers.\\
			  e.g. $gcd(21; 28) = 7$
		\item Least common multiple: $lcm(n; m)$\\
			  Computes the least common multiple of two integers.\\
			  e.g. $lcm(3; 4) = 12$
		\item Sinus / Cosinus: $sin(x)$, $cos(x)$\\
			  Computes the sinus / cosinus of an angle.\\
			  Don't forget to specify the input unit (otherwise I'll use radians).\\
			  The output unit is always radian.\\
			  e.g. $sin(90 \deg) = \sin \frac{1}{4} \tau = 1$
		\item Conversions:
			  \begin{itemize}
			  	\item $rad(x\;$deg$)$: Degree $\Rightarrow$ Radians
			  	\item $rad(x\;$grad$)$: Gradian $\Rightarrow$ Radians
			  	\item $rad(x\;$turn$)$: $\tau$-Radians $\Rightarrow$ Radians
			  	\item $deg(x\;[$rad$])$: Radians $\Rightarrow$ Degree
			  	\item $deg(x\;$grad$)$: Gradian $\Rightarrow$ Degree
			  	\item $deg(x\;$turn$)$: $\tau$-Radians $\Rightarrow$ Degree
			  	\item $grad(x\;[$rad$])$: Radians $\Rightarrow$ Gradian
			  	\item $grad(x\;$deg$)$: Degree $\Rightarrow$ Gradian
			  	\item $grad(x\;$turn$)$: $\tau$-Radians $\Rightarrow$ Gradian
			  \end{itemize}
		\item Iterations: $iterate(firstx: expression)$\\
			  Runs the given iteration, resulting the fix point.\\
			  e.g. $iterate(4: (xn+(9/xn))/2) = iterate(4: \frac{x_n+{\frac{9}{x_n}}}{2}) = \sqrt{9} = 3$\\
			  expresses $x_{n+1} = \frac{x_n+{\frac{9}{x_n}}}{2}$
		\item Short sum: $sum(var: start < end \;[$\$$\; step]; expression)$\\
		      Sums the expression several times, iterating $var$ from $start$ to $end$.\\
		      e.g. $sum(i: 1 < 4; i) = \displaystyle\sum_{i=1}^{4} i = 1+2+3+4 = 10$
		\item Short product: $product(var: start < end \;[$\$$\; step]; expression)$\\
			  Multiplies the expression several times, iterating $var$ from $start$ to $end$.\\
			  e.g. $product(i: 1 < 5; i) = \displaystyle\prod_{i=1}^{5} i = 5! = 1\times2\times3\times4\times5 = 120$
		\item Zero: $zero(var: expr)$\\
			  Finds the zero of a function.\\
			  e.g. $zero(a: 5*a-55) = 11$\\
			  because $5\times11-55=0$
		\item Solve: $solve(var: expr1 = expr2)$\\
			  Solves the equation.\\
			  e.g. $solve(a: a\;\hat{}\;2=49) = 7$\\
			  because $7^2=49$
	\end{itemize}
	
	\subsection{Constants}
	
	Fu Bar knows these constants:
	
	\begin{itemize}
		\item tau: $\tau$, a full radian turn, about 6.28
		\item pi: $\pi$, $\frac{\tau}{2}$, a half radian turn, about 3.14
		\item phi: $\Phi$, the golden ratio, about 1.62
		\item full: A whole, 1
		\item half: $\frac{1}{2} = 0.5$
		\item quarter: $\frac{1}{4} = 0.25$
	\end{itemize}
	
	\subsection{Units}
	
	\begin{itemize}
		\item deg: Degrees, full turn = 360
		\item grad: Gradians, full turn = 400
		\item rad: Radians, full turn = $\tau$
		\item turn, tau: $\tau$-Radians, full turn = 1
		\item pi: $\pi$-Radians, full turn = 2
	\end{itemize}
	
	\subsection{Comparison}
	
	You can simply compare two expressions using an equation operator:\\
	$expression_1 = expression_2$\\
	\\
	Fu Bar will tell you the result, and whether the terms are equal or not.\\
	If not, it will tell you, which term is the greater one, and output the difference and the factor.
	
	\subsection{Custom variables}
	
	You can simply save results into your own variables:\\
	$[\,name\,]\;expression$\\
	\\
	Fu Bar will compute the term and save the result to a variable of your choice.\\
	To mark it as a constant, insert an exclamation mark before the name:\\
	$[\,!name\,]\;expression$\\\\
	Note: Fu Bar won't overwrite constants!
	
	\subsection{Custom functions}
	
	Of course, you can define custom functions, too:\\
	$[\,name(\,var\,)\,]\;expression$\\\\
	To mark it as read-only, insert an exclamation mark before the name:\\
	$[\,!name(\,var\,)\,]\;expression$\\\\
	To ensure, that the argument is not zero, add 'nonzero' before the var name:\\
	$[\,name(\,$nonzero$\,var\,)\,]\;expression$\\\\
	To assign a separate value to $name(0)$, enter:\\
	$[\,name(0)\,]\;expression$\\\\
	For example:\\
	\begin{tabular}{ll}
	$[\,f(x)\,]\;25x$ & $\Rightarrow 0$\\
    $f(4)$& $\Rightarrow 100$\\
    \\
    $[\,f($nonzero$\,x)\,]\;1/x$ & $\Rightarrow 1$\\
    $f(4)$ & $\Rightarrow 0.25$\\
    $f(0)$ & $\Rightarrow$ Function not defined for $x=0$\\
    \\
    $[\,f(0)\,]\;0$ & $\Rightarrow 0$\\
    $f(4)$ & $\Rightarrow 0.25$\\
    $f(0)$ & $\Rightarrow 0$
	\end{tabular}
	
	\subsection{Explanations}
	
	Fu Bar can tell you the steps, how to calculate a term.\\
	$:explain\;32*(1+5)$ will result:\\
	$1\,+\,5\,=\,6$\\
	$32\,*\,6\,=\,192$\\\\
	Come on, try it!

\end{document}